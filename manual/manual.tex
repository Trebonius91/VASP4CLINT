\documentclass[a4paper,11pt]{article}
\usepackage[utf8]{inputenc}
\usepackage[english]{babel}
\usepackage{booktabs} % f\"ur Tabellen (Linien)
\usepackage{subfig}
\usepackage{graphicx} % f\"ur Bilder
\usepackage{import} % f\"ur pdf Bilder in einem anderen Ordner
%\usepackage{cite} % für die Lit.bib
\usepackage{amssymb} % f\"ur Mathe Symbole
\usepackage{amsmath} % f\"ur Matrizen
\usepackage[usenames,dvipsnames]{xcolor} % f\"ur farbe
\usepackage{colortbl} % f\"ur farbige zeilen in tabellen
\usepackage{microtype} % bessere schriftsetzung f\"ur pdflatex
\usepackage{tikz}
\usepackage{notoccite} % "`richtige"' cite reihenfolge
\usepackage{chemfig} % paket für chemische Summen- und Strukturformeln
\usepackage{placeins}
\usepackage{float}
\usepackage{verbatim}
\usepackage{afterpage}
\usepackage{setspace} % Zeilenabstand
\usepackage{algorithm}
\usepackage{hyperref}
\usepackage{algpseudocode}
\usepackage{pdfpages}

\usepackage{cite}
\usepackage{notoccite}

\usepackage{xcolor}
\hypersetup{
    colorlinks,
    linkcolor={red!50!black},
    citecolor={blue!50!black},
    urlcolor={blue!80!black}
}

\onehalfspacing % 1.5x
\usepackage[inner=2cm,outer=2cm,top=2cm,bottom=2cm,includeheadfoot]{geometry} % Seitenr\"anderen
%opening
\title{VASP4CLINT}
\author{Julien Steffen}

\begin{document}

\maketitle

This tutorial sheet is designed to be a tool for better remembering of VASP settings and/or
other preparation/evaluation routines and scripts used within the calculation
of CLINT systems (liquid surface catalysis). Most calculations, however, can be
applied for all types of surface (or even bulk) systems.
For internal use only! No responsibility is taken for the
correctness/completeness! For more detailed information, visit the VASP wiki (\url{https://www.vasp.at/wiki/index.php/The_VASP_Manual}).

\tableofcontents

\section{Description of Input Files}

Here, a short introduction is given to the general contents and layouts of the 
four main input files that are needed for every VASP calculation (depending 
on the type of calculation, other files might be needed as well, see below).
These are \texttt{INCAR} (containing keywords), \texttt{POSCAR} (containing the 
atomic coordinates and unit cell geometry of the system), \texttt{POTCAR} (containing 
the PAW potentials for the description of atomic cores) and \texttt{KPOINTS} (containing 
the setup for the integration of reciprocal space).
The given files are for an example single point calculation of a SCALMS surface slab
with an ethylene molecule adsorbed at the surface, their general structure can however 
be translated to most other systems.

\subsection{INCAR}

The keywords for the calculation must be given as upper case identifiers, followed 
by a space, an equal sign, a space, and the value of the variable: \texttt{WORD = VALUE}.
It is advisable to add most of the following keywords to each calculation (although there 
exist default values for most of them, if they are not listed in the \texttt{INCAR} file)
and think about their values for the current system.

\begin{itemize}

\item \texttt{SYSTEM = [Name]}  Identifying name for the calculation
\item \texttt{PREC = Normal}   Sufficient for geometry optimizations and molecular dynamics trajectories.
   For energy calculations of high quality, DOS, band structure or STM calculations, \texttt{PREC = Accurate}
   should be used.
\item \texttt{GGA = PE} The usual PBE GGA DFT functional (used in most cases, maybe
hybrid or RPA for $\Delta$-machine learning?)
\item \texttt{NCORE = 18}  Number of cores per orbital, should be a good choice for
highly parallelized calculations.
\item \texttt{ISPIN = 1}  If low concentrations of magnetic elements in Ga or similar solvents 
are calculated (quenching), or no open shell systems are present at all, no spin polarization
should be enough. However, electronic
convergence during Machine Learning can sometimes be improved by setting it to 2 (or if unknown systems
shall be tested for spin polarization). If bond breakings or open shell systems are treated, spin
polarized calculations should be done as well. If it is not certain´ whether spin polarization is needed,
one can do a spin polarized single point calculation and check the magnetic moment (written to \texttt{vasp.out}).
If its (almost) zero, \texttt{ISPIN = 1} should be sufficient.
\item \texttt{MAGMOM = [natoms]*1.0}  If \texttt{ISPIN=2} is set, for each atom, an initial magnetic 
moment should be given. The format must be written with explicit * sign, where the atoms in 
the \texttt{POSCAR} must be covered all (e.g. 40*1.0 for 40 atoms). For more complicated setups, the magnetic 
moments might be set by chemical intuition or by looking at literature references.
\item \texttt{NELM = 300}   Maximum number of electronic SCF steps, should be sufficient (if so 
convergence is reached within this limit, something else might have gone wrong).
\item \texttt{NELMIN = 4}  The minimum number of electronic SCF steps; might be raised if geometry 
optimizations cause problems, e.g., to 8.
\item \texttt{ENCUT = 400}   The plane wave cutoff in eV. For static calculations, it is
sufficient to choose the largest \texttt{ENMAX} value of all elements in the \texttt{POTCAR} file.
If cell parameters are relaxed, the \texttt{ENCUT} value should be 25-30\% higher.
For geometry optimizations of complicated surface adsorbates, it proved be useful as well
to raise the \texttt{ENCUT} value, e.g., to 600 eV for SCILL systems.
\item \texttt{EDIFF = 1E-07}  The electronic convergence criterion, energy difference in eV between 
two cycles. Might be lowered to 1E-06 (1E-05 is VASP default)
\item \texttt{LREAL = Auto}   Evaluation of projection operators in real or reciprocal space, 
this option should be the best for systems with more than 30 atoms. For small systems, \texttt{LREAL =  .FALSE.} can be used. VASP will give a warning, if the used scheme is not optimal for the
chosen system.
\item \texttt{ALGO = Fast}   Which kind of SCF algorithm is used (Davidson or DIIS).
For molecular dynamics with SCALMS systems, \texttt{Very Fast} will be enough, in most
other cases, \texttt{FAST} is well suited (Davidson for first steps of SCF cycle, then
switching to DIIS). If convergence issues appear, one might try \texttt{ALGO = Normal},
where only Davidson is used.
\item \texttt{ISMEAR = -1}  The chosen smearing method too enable the integration of bands.
The choice of the correct smearing scheme is not trivial, details can be found in
the VASP wiki\footnote{\url{https://www.vasp.at/wiki/index.php/Available_PAW_potentials}}.
In general, it can be said that \textbf{Fermi smearing} (\texttt{ISMEAR = -1}) can be used for
molecular dynamics (especially of SCALMS systems), \textbf{Methfessel-Paxton smearing first
order} \texttt{ISMEAR = 1} should be used for metal bulk or metal surface systems
(with and without adsorbates), \textbf{Gaussian smearing} (\texttt{ISMEAR = 0}) should be used
for insulators or single molecules and all other systems of unclear electronic
properties, and the \textbf{tetrahedron smearing} with Blöchl corrections (\texttt{ISMEAR = -5}) should be used for highly
accurate single point calculations like DOS calculations (but not for band structures!).
\item \texttt{SIGMA = 0.06}  Width of the smearing in eV. Fermi: 0.05-0.07 (temperature-dependent),
Gaussian: 0.03-0.05, Methfessel-Paxton: 0.1-0.2, tetrahedron: no \texttt{SIGMA} needed/ignored.
\item \texttt{IVDW = 12}  Grimme D3 dispersion correction, should usually be turned on for surface-adsorbate interactions etc., essentially no overhead
\item \texttt{LWAVE = .FALSE.} No wavefunction written to \texttt{WAVECAR}. Should be the default 
in order to limit memory pollution. \texttt{WAVECAR} files are needed, e.g.,  for STM simulations and implicit solvent calculations.
\item \texttt{LCHARG = .FALSE.}  No Charge densities are written to \texttt{CHGCAR} and \texttt{CHG}. Also 
default for memory saving, only needed for band structures/ Bader charges, STM pictures, charge
densities, etc.
\item \texttt{ISYM = -1}  No symmetry will be used in calculations, useful in liquid SCALMS systems,
needed for frequency calculations (see below).
\end{itemize}


\subsection{POSCAR}


In the following, a \texttt{POSCAR} file for a SCALMS system is shown.

  \begin{verbatim}
  Ethylene on GaxNi
   1.00000000000000
    16.8000000000000007    0.0000000000000000    0.0000000000000000
     0.0000000000000000   16.8000000000000007    0.0000000000000000
     0.0000000000000000    0.0000000000000000   32.0000000000000000
   Ni   Ga   H    C
     1   179     4     2
Selective
Direct
  0.4401850720183311  0.5238825428341128  0.3521766561531592  T T T
  0.1937295231910164  0.9105475427435126  0.1288196838077396  F F F
  0.9672328401478797  0.0562122444305433  0.0952092871570915  F F F
  0.9877192681027167  0.0507876003911017  0.1908843125036520  F F F
 ....
  0.5432764891825791  0.5384129105162143  0.4704130883460799  T T T
  0.5431780772189526  0.4585775872518857  0.4708784724342397  T T T
  \end{verbatim}
  
The first line is always a \textbf{comment}, which can be chosen arbitrarily
The second line contains a \textbf{global scaling factor}, which will automatically
be applied on the unit cell vectors and cartesian coordinates (if used).
Usually it is advisable to keep it at unity, since direct manipulation
of the \texttt{POSCAR} file by text editing is much harder if a scaling
factor must be considered.

For SCALMS surface slab systems, the unit cell will usually be an orthorhombic one,
with 90 degree angles,
but axis of arbitrary length (where x and y will usually be of equal length 
as well). The slab is cut in z-direction, where 15-20 Angstrom vacuum will be 
included (in order to keep electronic interactions between copies in z-direction
negligible).

Then, the list of elements follows. Atoms always should be ordered by element (else,
the associated \texttt{POTCAR} file would become unnecessarily large.
The numbers in the seventh line indicate, how many atoms of each element are present 
in the \texttt{POSCAR} file. The atoms are listed in the order as their element names are given in the 
names line.  Here one Ni atom is followed by 179 Ga atoms, 4 H and 2 C atoms. 

If \textit{selective dynamics} is activated, the keyword \texttt{Selective} must follow 
in line eight. With this, each degree of freedom in the system can either be frozen or 
movable. Frozen degree of freedom (a,b or c axis) are indicated with a \textit{False} flag (\texttt{F}), 
movable ones with a \textit{True} flag (\texttt{T}), where the flags must follow the coordinates of 
each atom. If no Selective keyword is present, the flags will be ignored.

The \texttt{Direct} keyword indicates that \textit{direct coordinates} are used, i.g., the coordinates 
are expressed as fractions of the unit cell vectors given in lines 3 to 5.
Alternatively, the \texttt{Cartesian} keyword can be given, where all atomic coordinates are 
given in \AA in usual \textit{Cartesian coordinates}.
If no selective dynamics shall be used, this keyword must be placed in line eight.

How can a \texttt{POSCAR} file be set up? 
For crystalline systems like metal surfaces, it is advisable to take a unit cell structure 
from literature or write one by hand. Then, the \texttt{p4v} program can be used to multiply it.
Simply open the \texttt{POSCAR} file of the unit cell:

\begin{verbatim}
 p4v POSCAR
\end{verbatim}

Then, go to \textit{Edit $\rightarrow$ Multiply Cell} and enter the desired replication in all three
coordinate axes. 
Further, one can click together a usual xyz file with a program like \texttt{Avogadro} and convert it 
to a \textit{POSCAR} file with the \texttt{xyz2poscar} script (section \ref{xyz2poscar}).

For special systems, further utility scripts exist: If a SCALMS random alloy shall be built, the 
\texttt{build\_scalms.py} script (section \ref{build_scalms}) might be used for the initial setup.
If adsorbates shall be placed on surfaces, the \texttt{build\_adsorbates.py} script (section \ref{build_adsorbates})
is well suited.
In some cases, the most interesting part of the unit cell is not in its center and therefore
its visualization gets more complicated. Then, the script \texttt{shift\_unitcell.py} 
(section \ref{shift_unitcell}) might be used 
to move the content of the cell along its axes (the Physics stays the same!).
For the buildup of inclined surface cells needed for some adsorption patterns, the \texttt{cut\_unitcell}
program (section \ref{cut_unitcell}) might be used.

Further, conversion between direct and Cartesian coordinates must be done in some cases (if, e.g., the
width of the vacuum above a surface slab shall be enlarged by making the c-axis longer, 
this can only be done in Cartesian coordinates;
else, all atoms would be shifted uniformly). This can be done with the \texttt{p4v} program as well.
Open the \texttt{POSCAR} and click on \texttt{Build} on the left side. There, select either 
\textit{direct} or \textit{Cartesian} in the check boxes.


\subsection{POTCAR}       
       
Here, the PAW potentials for the elements appearing in \texttt{POSCAR}
 are written in a text file, directly concatenated. 
In order to build up a new file, first, a folder with \texttt{POTCAR}s should be generated (e.g.
in the home or scratch folders). 
Then, the new \texttt{POTCAR} will be built up like this (here shown for the example \texttt{POSCAR}
above).
       \begin{verbatim}
touch POTCAR
cat /scratch/potcar/PAW_PBE.52/Ni/POTCAR >> POTCAR
cat /scratch/potcar/PAW_PBE.52/Ga/POTCAR >> POTCAR
cat /scratch/potcar/PAW_PBE.52/H/POTCAR >> POTCAR
cat /scratch/potcar/PAW_PBE.52/C/POTCAR >> POTCAR
       \end{verbatim}
       Always use \texttt{cat} (or similar commands) instead of copying the original files,
       since those are write-protected/owned and might lead to problems when copying them to a computing 
       cluster/renaming or similar things.
       For most elements, different \texttt{POTCAR} file are available. Which one is 
       recommended by the VASP developers can be seen on the 
       page\footnote{\url{https://www.vasp.at/wiki/index.php/Available_PAW_potentials}}.
   For SCALMS systems, e.g., the smaller version of Ga \texttt{POTCAR} with three explicit 
   electrons is usually used.
   
   The \texttt{POTCAR} file, once generated, can usually be treated as black box. One important exception is 
   the determination of the correct \texttt{ENCUT} value in the \texttt{INCAR}. Execute
   
   \begin{verbatim}
     grep ENMAX POTCAR
   \end{verbatim}

   and the \texttt{ENMAX} values for all involved elements are printed out. If the shape of the 
   cell is kept and no highly involved geometry optimization is done, the largest value 
   might be taken as \texttt{ENCUT} value (which is done if no \texttt{ENCUT} is given in the 
   \texttt{INCAR} at all). Else, the value might be the reference for, e.g., 25-30\% increase, when 
   cell parameters are relaxed (see above).
       
\subsection{KPOINTS}

The file with information about the reciprocal space integration in periodic
lattices.
First, one should consider in which direction the electronic structure of the system 
actually is periodic. 
For single molecules in an empty cell, only the $\Gamma$ point must be sampled. Then, the 
\texttt{KPOINTS} file has the following appearance:

\begin{verbatim}
K-Points
0
Gamma
1  1  1
0  0  0
\end{verbatim}
       
If, e.g., for the SCALMS surface slab shown above, vacuum is introduced in the c direction,
only one k-point should be calculated there, and more than one in other directions, e.g., 
a 2x2x1 grid for a cheap MD simulation:
       
\begin{verbatim}
K-Points
0
Gamma
2  2  1
0  0  0
\end{verbatim}

For our purpose, the grid should always be centered at the $\Gamma$ point! (keyword \textit{Gamma}
in the third line).
The number of k-points in each coordinate should be thoroughly benchmarked if a completely new 
system is treated (e.g, by comparing energies or band structures).

In general, the number of k-points per axis is inversely proportional to the length
of this axis. If, e.g., the x axis is 20 \AA~long and the y-axis only 7 \AA~long, a
good choice might be 2x6x1 k-points (if we still look at a surface-slab).



\section{Geometry Optimization}

For geometry optimizations, all settings from above can be used. In addition, some others should be added.
Unfortunately, VASP has a really bad geometry optimizer based on Cartesian coordinates,
therefore, it can be very tedious to converge a geometry within a good convergence criterion, 
especially if large, periodic systems (molecules on surfaces etc) are used.
A good tip might be to set as many atoms as possible to frozen ones in the \texttt{POSCAR} file 
(activating Selective Dynamics and write \texttt{F F F} near the respective atoms), since then 
the dimensionality of the optimization search space will be reduced. Further, convergence might 
occur after a large number of ionic steps, more than 200-300 (up to thousands in some cases)
might be needed. For those very long calculations, it might be advisable to use the 
\texttt{vasp\_long.sh} script in order to automatize the restart after the walltime has 
been exceeded (section \ref{vasp_long}).
As mentioned above, the \texttt{NELMIN}-value might be raised if the optimization is stuck at essentially
the same structure and electronic SCFs require only 2-4 steps. The accuracy's (\texttt{EDIFF},
\texttt{PREC}, \texttt{ALGO} and \texttt{ENCUT}) might also be raised in order to enforce convergence.
Important keywords:

\begin{itemize}
\item \texttt{IBRION = 1}  Activates the quasi-Newton geometry optimization algorithm. 
Usually the "best" one, but requires a reasonable starting structure, else, explosions might occur. 
\texttt{IBRION = 2} activates the simpler conjugate-gradient algorithm, might be used for bad starting structures, 
but converges even more slowly. Maybe first 2, then 1?
\item \texttt{ISIF = 0/2} Optimization of atom positions within the fixed unit cell. The usual approach 
for SCALMS systems. (in 2, the stress tensor is calculated as well, 0 saves some time).
For metal surfaces, it is advisable to first optimize the shape of a metal bulk unit cell
(\texttt{ISIF = 7}), and generate a metal surface slab unit cell using the optimized
atom distances. No cell volume optimization should be done for the surface slab itself!
\item \texttt{EDIFFG = -0.02} The criterion for geometric convergence. A value often used in publications 
is 0.02 or 0.01 eV/A for the maximum gradient component, such that all components of the actual gradient 
for the atoms activated for optimization (\texttt{T T T} with selective dynamics) must be
below this value. For larger systems, 0.03 might be OK (reported in the literature) as well. Alternatively, 
one might use the change in total energy as convergence criterion (e.g., \texttt{EDIFFG = 1E-4}), now without a 
minus sign. Then, convergence is reached if the total (sigma $\rightarrow$ 0) energy between two geometry optimization 
steps is lower than this value. If convergence cannot be reached within the force criterion and all 
other methods fail, one might use for example \texttt{EDIFFG} = 1E-4 (the default value in VASP), even if the 
force-criterion seems to have a better reputation in literature.
\item \texttt{NSW = 200} Number of "ionic" steps, i.e., geometry optimization steps. Usually, quite a 
large number is needed (even thousands of steps), but sometimes unexpectedly fast convergences might occur.
\item \texttt{POTIM = 0.5}  The width of a geometry optimization step. One of the main opportunities to 
fine-tune the optimization if convergence problems arise.
\item \texttt{NFREE = 15}  Number of remembered ionic steps for approximate buildup of the Hessian matrix
within the quasi-Newton algorithm. 10-20 is the recommended value for rather large systems in the VASP Wiki.
\item \texttt{ISYM = -1}  No symmetry will be used in calculations, useful for systems with
no ordering or symmetry. For clean metal surfaces, symmetry should be activated (\texttt{ISYM = 2}).
\end{itemize}


\section{Single Point Calculation}

If one only wants to calculate the energy of a system (e.g., a converged geometry for a minimum or 
intermediate) with high precision, some minor changes might be made.
The number of k-points should be raised in comparison to the geometry optimization (e.g., 4x4x1 instead of 2x2x1).
The final energy can be found in the \texttt{OUTCAR} file:

\begin{verbatim}
   FREE ENERGIE OF THE ION-ELECTRON SYSTEM (eV)
  ---------------------------------------------------
  free  energy   TOTEN  =      -956.59244435 eV

  energy  without entropy=     -956.33674483  energy(sigma->0) =     -956.50721118
\end{verbatim}

The reported energy should always be the one with sigma extraploated to zero
(\texttt{energy(sigma->0)}), in this case:  -956.50721118 eV.

List of changed keywords:

\begin{itemize}
\item \texttt{ISMEAR = -5}  Activates the Tetrahedron Smearing with Blöchl corrections, the best one, if only energies are needed (or \texttt{ISMEAR = 0} with a very small \texttt{SIGMA} (e.g. 0.01)).
\item \texttt{PREC = Accurate} The overall precision (of the FFT) might be raised.
\item \texttt{IBRION = -1}  No positional updates
\item \texttt{NSW = 0}   The structure is calculated only once.
\end{itemize}

\section{Ab Initio Molecular Dynamics}

An important tool for the analysis of CLINT systems is molecular dynamics, since static
single points will be usually of less interest in the description of liquid phase 
systems.
Long trajectories can be started with the \texttt{md\_long.sh} script (section \ref{md_long}).
The most important VASP keywords for that purpose are (in addition to the general
keywords from above):

\begin{itemize}
 \item \texttt{IBRION = 0} Activates the MD routine
 \item \texttt{MDALGO = 2} The Nose-Hoover thermostat (deterministic) will be used,
 the initial velocities will be generated by chance (unless they are read in from 
 a previous \texttt{CONTCAR} file copied to \texttt{POSCAR}, where the velocities 
 of the atoms are written below the coordinates).
 \item \texttt{NSW = 1000} The number of time steps. Depending on the size of the system, a step
 might take much time, so the duration for taking a single MD step might be tested first
 within a short MD calculation (but rather 10 than 1 steps, since the first step always takes 
 longer, as the whole electron density and wavefunction must be built from scratch there).
 \item \texttt{POTIM = 10.0} The MD-time step in fs. For pure SCALMS systems containing only
 metal atoms, a quite large step of 10 fs should work well. If, however, hydrocarbon
 molecules or similar nonmetallic compounds are simulated in the same cell, the step should be 
 much smaller, maybe 1-2 fs, depending on the temperature, since hydrogen atoms move/vibrate much
 faster!
 \item \texttt{TEBEG = 700} The initial temperature in K.
 \item \texttt{TEEND = 700} The final temperature in K. If \texttt{TEBEG} has another value,
 the temperature will vary linearly during the dynamics from the start to the end value.
 \item \texttt{SMASS = 0} The mass of the Nose-Hoover temperature degree of freedom. Depending 
 on the value, the temperature of the system will vary more or less around the ``ideal''
 value given by the user. If the Nose-Hoover oscillator is too stiff, however, the
 ergodicity of the system might be harmed! Therefore, the default value should usually 
 be good (40 time steps), this is set by the 0 flag.
 \item \texttt{IWAVPR = 12} How the new orbitals are predicted from the old ones. For MD calculations,
 VASP strongly recommends 12. Might lead to problems for Machine learning calculations (see below)!
 \item \texttt{LMAXMIX = 4} PAW charge densities from up to d-electrons are passed through
 the charge-density mixer. Should be good for SCALMS (and SCILL) systems.
 \item \texttt{MAXMIX = 100} Number of previous steps stored in the Broyden charge mixer for 
 subsequent MD steps. Might be commented out, probably deprecated (?).
\end{itemize}

\section{Machine Learning Force Field}

The new Machine Learning Force Field (ML-FF) within VASP allows for the on-the-fly generation of force fields from
AIMD trajectories (however, geometry optimizations and frequency calculations can be done as well!). 
With this, larger systems can be simulated over longer time scales.

\subsection{On the Fly Learning}

The input for such a calculation is similar to that for an AIMD simulation (see last section).
First, set up the usual AIMD simulation and then add some keywords to the \texttt{INCAR} file:

\begin{itemize}
 \item \texttt{ML\_LMLFF = .TRUE.} Activates the ML-FF (must be set for all ML-FF calculations, 
    learning as well as application). 
 \item \texttt{ML\_ISTART = 0} Activates the learning from scratch, i.e., without ab-initio data from 
    former runs (see below successive learning). For VASP 6.4: \texttt{ML\_MODE = TRAIN} (no \texttt{ML\_AB} file 
    shall be present at the beginning!).
 \item \texttt{ML\_MB = 3000} This is the number of basis functions per element. Especially if light
    elements like hydrogen are involved or many different elements in complicated molecules like ILs 
    shall be learned, this number should be much higher than the default of 1500. If the value is
     too large, however, the memory of the machine might be too small and errors occur.
 \item \texttt{ML\_MCONF = 2000} The number of collected reference configurations. This is the total number
    of DFT calculations to be done during 
    the learning. The default of 1500 might be too small for systems with many elements. If the value is
     too large, memory issues will occur (see above).
 \item \texttt{ML\_LBASIS\_DISCARD = .TRUE.} The calculation is not aborted, if the total number of basis 
   functions (\texttt{ML\_MB}) is reached. This is very important if the systems to be learned are complicated. 
   Then, the largest possible \texttt{ML\_AB} value (before the memory limit is reached) might be too small 
   to get all basis functions. Instead of aborting, former basis functions/configurations will be
   replaced by new ones.
   It should be tested whether the quality of the overall ML-FF could suffer from this strategy. So far, however, 
   it is the only viable one. 
 \item \texttt{ML\_RCUT1 = 5.0} The default cutoff value for the distance descriptor in \AA.
   It might be useful to raise this value if interactions over more than 5~\AA~are important. In reality, 
   this is of course almost always the case, since Coulomb interactions decay much slower. Longer cutoff values 
   will increase the complexity and memory requirements of the force field (larger ML\_MB, ML\_MCONF values), therefore,
   some compromise should be found.
 \item \texttt{ML\_RCUT2 = 5.0} The cutoff value for the angle (or three-body) descriptor in \AA. Should be similar to
   the value of \texttt{ML\_RCUT1} (further tests/benchmarks required).
 \item \texttt{TEBEG = 300} It is recommended by the VASP developers to gradually heat the system during the learning.
   Further, the maximum learning temperature should roughly be 30\% larger than the maximum temperature occurring during 
   the application of the force field.  
 \item \texttt{TEEND = 950} Former keyword continued: If the application temperature is, e.g., 700 K, the end temperature
   during the learning should be at least 30\% higher, for example 950 K. With this, the system is heated up from 
   300 K to 950 K during the learning trajectory. The number of time steps should be large enough to enable a 
   realistic heating rate.
\end{itemize}

Besides these keywords, the \texttt{MAXMIX} keyword (added for usual AIMD simulations) should be \textbf{removed}
from the \texttt{INCAR} file! Since hundreds or thousands of ML-FF steps might be calculated between two DFT single points,
the Broyden charge mixer of subsequent frames is no longer useful and SCF problems might occur.

After the learning process, a \texttt{ML\_ABN} file containing all collected DFT configurations in a human-readable 
format and a \texttt{ML\_FFN} file containing the actual machine learning force field (binary format) are written.
The \texttt{ML\_ABN} file should be renamed to \texttt{ML\_AB} for subsequent learning (next subsection).
The \texttt{ML\_FFN} file should be renamed to \texttt{ML\_FF} and can be used for further simulations etc.

A special case is the \textbf{learning in bulk liquids}. There, it is usually advisable to use NpT dynamics with variable volume,
since the density of the liquid will change upon heating. 
For this, the following settings must be added:

\begin{itemize}
 \item \texttt{MDALGO = 3} Activates the Langevin thermostat/barostat
 \item \texttt{LANGEVIN\_GAMMA = 5.0 5.0} Value of friction coefficients for thermostat, one value per element in the cell!
 \item \texttt{LANGEVIN\_GAMMA\_L = 5.0} Friction coefficient for lattice degrees of freedom 
 \item \texttt{PMASS = 1000} Fictitious mass for lattice degrees of freedom
\end{itemize}

Usually, it is desirable to keep the initial shape of the periodic box during a NpT simulation (only breathing allowed).
In order to enforce this, a \texttt{ICONST} file 
must be added to the folder, containing the following lines:

\begin{verbatim}
LA 1 2 0
LA 1 3 0
LA 2 3 0
LR 1 0
LR 2 0
LR 3 0
S  1  0  0  0  0  0 0
S  0  1  0  0  0  0 0
S  0  0  1  0  0  0 0
S  0  0  0  1 -1  0 0
S  0  0  0  1  0 -1 0
S  0  0  0  0  1 -1 0
\end{verbatim}

After learning, the (expected) \textit{quality} of the force field can directly be analyzed by postprocessing the \texttt{ML\_LOGFILE} file.
One can for example print out the prediction errors of energy and force components:

\begin{verbatim}
 grep ERR ML_LOGFILE
\end{verbatim}

\subsection{Subsequent Learning Lessons}

In many cases, a single on the fly learning simulation will not be enough. These cases involve either: (a) complex 
or large systems, where several days of calculations are needed until convergence, too long for the maximum walltimes on clusters
like Fritz, or (b) several different parts/subunits of a system shall be learned one after another, e.g., a metal surface, 
a liquid bulk, and interactions between surface and liquid molecules.

For \textbf{case (a)}, the second (and further) calculation need to have some changes in the \texttt{INCAR} file:

\begin{itemize}
 \item \texttt{ML\_ISTART = 1} The learning is continued from an existing \texttt{ML\_AB file}. For VASP 6.4: \texttt{ML\_MODE = TRAIN} (with the \texttt{ML\_AB} file from the first part present in the folder).
\item \texttt{ML\_CTIFOR = [value]} Sets the threshold for the Bayesian error estimation. This parameter needs to be set,
 else, many DFT calculations will be done at the beginning, similar to the first part of the learning. This is usually not needed 
 since we are continuing a simulation that already gathered some information about the area in configuration
 space in which we are currently located.
 The value of \texttt{ML\_CTIFOR} can be found out by analyzing the former part of the learning with the following command:
 \begin{verbatim}
  grep THRUPD ML_LOGFILE
 \end{verbatim}
 Then, a number of lines will be printed out, like this:
 \begin{verbatim}
  .....
 # THRUPD ##################################################### ...
 # THRUPD            nstep      ctifor_prev       ctifor_new    ... 
 # THRUPD                2                3                4    ...
 # THRUPD ##################################################### ...
 THRUPD                 13   2.00000000E-03   3.46063871E-02    ...
 THRUPD                 26   3.46063871E-02   3.11928675E-02    ...
  ....
 THRUPD             434606   2.33277435E-02   2.29238666E-02    ...
 THRUPD             459015   2.29238666E-02   2.30776577E-02    ...
 THRUPD             466213   2.30776577E-02   2.31027590E-02    ...
 \end{verbatim}
 The last \texttt{ctifor\_new} value should be taken, in this case: \texttt{ML\_CTIFOR = 2.31027590E-02} for the next run.
 \end{itemize}

 For \textbf{case (b)}, \texttt{ML\_ISTART = 1} must be set as well. Since we are now in a new subsystem, it is no
 longer useful to set \texttt{ML\_CTIFOR} to some value. It should not be added at all.
 
 Case (a) can now be handled in an automated way with the script \texttt{ml\_long.sh} (section \ref{ml_long}).
 With this, an arbitrary number of 
 learning lessons on a given system can be concatenated. The update of the \texttt{ML\_CTIFOR} as well as \texttt{TEBEG} values
 (if the system is heated during the learning) is done automatically.

 
\subsection{Accurate Force Fields}

The standard accuracy of the ML-FFs can be raised with certain settings. This will of course also raise the number 
of reference configurations and basis functions, thus, the memory requirement. 

In addition to the on-the-fly machine learning settings given above, add the following keywords:

\begin{itemize}
 \item \texttt{ML\_IALGO\_LINREG = 1} Selects the Bayesian linear regression for the learning process.
 \item \texttt{ML\_SION1 = 0.3} Decreases the width of a radial descriptor basis function (from 0.5 default)
 \item \texttt{ML\_MRB1 = 12} Increases the number of radial basis functions for the radial descriptor (from 8 default)
\end{itemize}

The latter two settings can of course be varied, such that more lor less broader or sharper basis functions are present.

After the usual learning, a \textbf{separate refit calculation} must be done!
For this, the following parameters should be added to the usual ML-FF parameters (\texttt{ML\_SION1} and \texttt{ML\_MRB1} 
must be kept at their values from above):

\begin{itemize}
 \item \texttt{NSW = 0} No MD time steps calculated (only static refitting)
 \item \texttt{MD\_IALGO\_LINREG = 3} Use singular value decomposition instead of Bayesian linear regression
 \item \texttt{ML\_CTIFOR = 1000.0 } Large value to force the refit.
 \item \texttt{ML\_EPS\_LOW = 1.0E-14} Threshold for sparsification of local reference configurations
\end{itemize}

Further, the \texttt{ML\_ABN} file from the previous run must be renamed to \texttt{ML\_AB} and copied to the calculation folder.
The refit should be finished after some minutes. The resulting \texttt{ML\_FFN} can be used as usual for ML-FF applications.

\subsection{Combining ML\_AB files}

If ML-FFs for complex systems like ionic liquids on metal surfaces shall be optimized, it can happen that the usual subsequent 
learning strategy (b) from above is not applicable, since the dynamics explodes for further parts/systems, even if no
\texttt{ML\_CTIFOR} value is net.
Further, it might be the case that ab initio data of, e.g., reaction paths sampled via steered MD has been collected in some 
way and shall now be included into the training set.

For this, \texttt{ML\_AB} files from separate calculations can be directly combined into one large \texttt{ML\_AB} file.
The following steps must be done:

\begin{enumerate}
 \item Select one of the \texttt{ML\_AB} files as the first one. This can stay as it is.
 \item In all other \texttt{ML\_AB} files, remove the header section until the lines:
 
 \begin{verbatim}
  **************************************************
     Configuration num.      1
 \end{verbatim}
 Before removing the header section, note the number of configurations, as well as the maximum number of 
 atom types, the atom types, the maximum number of atoms per system and the maximum number of atoms per atom type.
 
 \item Add the \texttt{ML\_AB} files together. For this, simply cat the following files after the first, e.g:
 
 \begin{verbatim}
  cat ML_AB2 >> ML_AB
 \end{verbatim}

 \item In the resulting large \texttt{ML\_AB} file, change the header section: Add up the number of configurations in all
  \texttt{ML\_AB} parts and write it in the fifth line.
  Determine the total number of different elements present in any of the configurations write the number into
  the ninth line. Write the symbols of all those elements into the 13th line (ordering does not matter). The maximum
  number of atoms in a system (in any of the configurations) should be written into line 17 and the maximum number 
  of atom type (again the largest value of all \texttt{ML\_AB} parts) should be written into line 21.
  

  \item  If the preparation of the \texttt{ML\_AB} file is finished, the \texttt{INCAR} file must be changed/amended somehow:
  \begin{itemize}
   \item \texttt{ML\_ISTART = 3} activates the generation of a new \texttt{ML\_FF} directly from a read-in \texttt{ML\_AB} file. 
   \item \texttt{NSW = 0} No MD steps shall be calculated
   \item \texttt{IBRION = -1}
   \item \texttt{ML\_MCONF = [number]} Must be the sum of all configurations in the new \texttt{ML\_AB} file!
   \item \texttt{ML\_LBASIS\_DISCARD = .TRUE.} Avoids dying of the calculation of the number of basis functions becomes too large
   \item \texttt{ML\_MCONF\_NEW = 12} Number of temporarily stored local configurations, seems to speed up the process
   \item \texttt{ML\_CDOUB = 4} Criterion of enforced DFT calculations, seems to speed up the process
   \end{itemize}

   
   \item Start the calculation. Depending on the number of local configurations and the complexity of the system, this can 
   take a considerable amount of time! Since this calculation cannot be restarted from intermediary, it might be necessary
   to do it on a cluster without (small) walltime limit.
\end{enumerate}

\subsection{Prepare a Fast ML-FF Calculation}

With VASP 6.4 it is now possible to perform much faster ML-FF simulations. For this, the \texttt{ML\_AB} file 
from the learning procedure must be postprocessed. This calculation might take minutes to hours, depending
on the complexity of the system and the number of reference configurations.

The \texttt{INCAR} file for this calculation needs to have the usual ML-FF settings, plus the following:

\begin{itemize}
 \item \texttt{NSW = 0} No MD time steps
 \item \texttt{ML\_MODE = REFIT} Activates refitting/generation of a fast ML-FF 
\end{itemize}

The resulting \texttt{ML\_FFN} file can than be used for fast calculations.

\subsection{Perform MD Simulations}

ML-FF MD simulations are similar to AIMD simulations. Therefore, all settings in the AIMD section can be reused here (all wave function
related keywords have of course no application and will simply be neglected by the program). Nevertheless, 
a \texttt{POTCAR} and a \texttt{KPOINTS} file must be present! Else, strange Fortran IO errors might occur that are hard 
to assign to a particular reason at first.
Further, the \texttt{ML\_FF} file must be there.

Concerning ML-FF specific keywords, followings needs to be present:

\begin{itemize}
 \item \texttt{ML\_LMLFF = .TRUE.} Activates ML-FF
 \item \texttt{ML\_ISTART = 2} No new learning but only calculation with the existing \texttt{ML\_FF} file. 
     For VASP 6.4: \texttt{ML\_MODE = RUN}.
 \item \texttt{ML\_MB = [number]} Number of basis functions per element, same as for learning!
 \item \texttt{ML\_MCONF = [number]} Number of local reference configurations, same as for learning!
 \item \texttt{ML\_RCUT1 = [value]} Radial cutoff, same as for learning (default: 5 \AA).
 \item \texttt{ML\_RCUT2 = [value]} Angular descriptor cutoff, same as for learning (default: 5 \AA)
\end{itemize}

For \textbf{accurate force fields}, the \texttt{ML\_SION1} and \texttt{ML\_MRB1} keywords must be set as well.

Usually, it is more useful to directly perform fast mode calculations (after the fast ML-FF was generated).
For this, additional keywords should be added:

\begin{itemize}
 \item \texttt{ML\_LFAST = .TRUE.} Activates the fast mode
 \item \texttt{ML\_OUTPUT\_MODE = 0} Less output is written, no radial distribution function is calculated.
 \item \texttt{ML\_OUTBLOCK = 100} Only each 100th step will be written out. This makes the calculation faster
 and smaller files are written. Of course only useful for certain observables (e.g. not for diffusion coefficients). One should decide for each calculations, how often geometries need to be written out.
\end{itemize}

Again, the \texttt{md\_long.sh} script (section \ref{md_long}) might be used to enable even longer MD trajectories.


\section{Normal Modes and IR Spectrum}

In order to compare calculations to experimental IR spectra, the Hessian matrix of the structure 
should be optimized. It is important that the structure is indeed a minimum (or a TS), else,
the quadratic Taylor assumption breaks down and the frequencies are meaningless!
Add the following keywords to the \texttt{INCAR} file:

\begin{itemize}
 \item \texttt{IBRION = 5} Activates a frequency calculation
 \item \texttt{POTIM = 0.015} The standard elongation for numerical Hessian calculations.
 \item \texttt{LDIPOL = .TRUE.} Activates the dipole moment corrections (for intensities
 of IR bands!)
 \item \texttt{IDIPOL = 3/4} For slabs, this should be 3, such that only contributions 
 in z-direction are considered (\textit{surface selection rule}, for isolated molecules,
 this should be 4.
 \item \texttt{DIPOL = 0.5 0.5 0.5} If \texttt{LDIPOL} is activated, it might happen that
 the electronic convergence is harmed and the whole calculation diverges quite fast with horrible errors.
 Then, the center of the cell might be shifted (in direct lattice coordinates) by try
 and error until the calculation runs through. In the example, the center is located
 indeed in the geometry center of the cell.
 \item \texttt{ISYM  -1} Switch off symmetry completely (symmetry will be broken anyway due to the small numerical
 elongations of the coordinates), especially needed if more than
 one node is used for the calculation! Else, strange errors (changing of k-point sets)
 will occur after the first single point.
\end{itemize}

For large systems (e.g., alkanes adsorbated on a SCALMS surface, one should use Selective 
Dynamics and only activate the atoms which vibrations are of interest in order to
save calculation effort!

After finishing the calculation, the script \texttt{vasp2molden.py} by Christian Neiss 
should be called in the calculation folder with the \texttt{vasprun.xml} file as argument (section \ref{vasp2molden}).
The file \texttt{vasprun.molden} is generated. 
This can be opened with the \texttt{gmolden} molecular viewer:

\begin{verbatim}
 gmolden vasprun.molden
\end{verbatim}

In \texttt{gmolden}, click on \texttt{Norm. Modes} in the upper right corner. An IR spectrum
is generated, the widths of the bands can be altered by the user. 
Select the respective mode on the list to view it in the molecule window.

\section{Implicit Solvation}

Adding explicit solvent molecules to an arbitrary system is possible with VASP.
Due to the scaling of DFT with the number of atoms, however, such calculations
will be very expensive and thorough sampling for, e.g., solvation free energies
is clearly out of reach.
An alternative, which at least is able to give some first glimpse concerning the
effect of solvents on a system is to use implicit solvation.

This is not implemented in VASP directly, instead, the \texttt{VASPsol}
package\footnote{\url{https://github.com/henniggroup/VASPsol}}
must be installed as an addon.

The installation process is well documented but not trivial, an already updated
VASP version on one of the calculation clusters might be used instead.

After installing, only single point energies can be calculated with solvation, no
gradients or frequencies!
For this, the \texttt{WAVECAR} file from a calculation of the same system must be
given. The calculation is therefore a two-step process:

\begin{enumerate}
 \item Do a single point calculation (with high accuracy etc.), and activate
 \texttt{LWAVE = .TRUE.}.
 \item Use the same settings and structure as for the first calculation, but add the
 following keywords:
 \begin{itemize}
  \item \texttt{ISTART = 1} The orbitals are read in from the \texttt{WAVECAR} file.
  \item \texttt{LSOL = .TRUE.} Activates the solvation calculation
  \item \texttt{EB\_k = 78.4} Relative permittivity of the solvent (here: water). The value
  for the desired solvent should be taken from literature.
 \end{itemize}

\end{enumerate}

 With these settings, the single point calculation can be redone and the energy difference
 between both calculations is the solvation energy.


 \section{Gibbs Free Energies}

 For the calculation of equilibria or rate constants of reactions, Gibbs free energies
 of intermediates or transition states must be calculated.
 When the textbook harmonic oscillator/rigid rotor approximation for the ideal gas
 is used, the free energy of a system can be essentially calculated from a
 frequency calculation.
 VASP itself is (once again) not capable of doing this.
 Instead, a Tool from Github (Python script) can be used:
 VASPGibbs\footnote{\url{https://github.com/ftherrien/VaspGibbs}}.

 The application is quite simple: First, do a frequency calculation of the
 structure (with or without selective dynamics is possible).

 Then, the analysis can be started with:

 \begin{verbatim}
  vasp_gibbs -t [temperatur]
 \end{verbatim}

 where the temperature, at which the reaction of interest takes place, must be given
 (e.g., \texttt{-t 300} for 300 K).

 The results of this analysis is written to the file \texttt{VaspGibbs.md}, which
 is quite well documented and somewhat self-explaining.

\section{External Electric Fields}

VASP has the ability to add external electric fields to a surface slab calculation.
For this purpose, a sawtooth potential is added to the cell. Since the kink
of the sawtooth is unphysical, it is only reasonable to add the electric field
to a surface slab system with vacuum between the layers, where the kink can be placed!

Electric fields can be added to single point calculations but also to, e.g., frequency
calculations for the determination of Stark shifts.

To the usual input of the calculation of choice, the following commands must be added:

\begin{itemize}
 \item \texttt{LDIPOL = .TRUE.} Switches on dipole corrections to the potential and forces in VASP.
 \item \texttt{IDIPOL = 3} The dipole correction is applied along the c axis (usual choice for slab systems)
 \item \texttt{EFDIELD = 0.4} Applies the electric field along the axis specified by \texttt{IDIPOL} with the given strength. Here, an electric field of 0.4 eV/\AA~is applied along the c axis.
\end{itemize}

When an external electric field is applied to a system, it might be interesting to know
how large is actual electric field is at a certain position within the simulation cell.
This can be done, when the \texttt{LVTOT = .TRUE.} command is added.

Then, the \texttt{LOCPOT} file is written (same format as the \texttt{CHGCAR} file), which
contains the local electrostatic potential in each integration grid point in the simulation
cell in eV.
In order to determine the effective electrostatic field at a certain region in space which
is introduced by the external electric field, a second \texttt{LOCPOT} file must
be calculated from a calculation of the same system \textit{without} applied electric field.

After subtracting the \texttt{LOCPOT} file of the calculation without the external electric
field from the \texttt{LOCPOT} file of the calculation with external electric field (grid point
per grid point!), the resulting file contains the local electrostatic potential change/difference introduced
by the external electric field. The change in the electric field can then be obtained by
calculating the numerical derivatives of that potential difference. See section \ref{calc_field_co} for an implementation.

\section{Density of States}

If density of states shall be calculated, the structure of the system should be optimized 
first (or taken from an MD trajectory).
Similar to the simple energy single point from above, the number of k-points should be 
increased.
In the \texttt{INCAR}, the following keywords must be added:

\begin{itemize}
 \item \texttt{LORBIT = 11} Activates the (possible) resolution of the DOS into 
 contributions from different orbitals and/or elements.
 \item \texttt{ISMEAR = -5} Again, the Tetrahedron smearing
 \item \texttt{NEDOS = 1001} Enhances the resolution (number of energies, for which the 
 DOS is calculated).
 \item \texttt{PREC = Accurate} High FFT precision should be used.
\end{itemize}

The \texttt{vasprun.xml} file should then be opened with \texttt{p4v}, where over
the \texttt{DOS+Bands} function, partial density of states for certain
elements or orbitals can be plotted and printed to file.

\section{Band Structures}

So far, band structure calculations for CLINT systems are scarce.
If they are needed nevertheless, the following strategy should be applied:

First, do a DOS (or single point) calculation with good quality and print out the 
\texttt{CHGCAR} file (\texttt{LCHARG = .TRUE.}). 

Then, modify the \texttt{INCAR} file of the DOS calculation slightly:

\begin{itemize}
 \item \texttt{ICHARG = 11} Density will be kept fix, no self consistency!
 \item \texttt{ISMEAR = 0} Gaussian smearing, tetrahedron smearing cannot be used for band structures!
 \item \texttt{SIGMA = 0.01} Small value needed
\end{itemize}

In the next step, the \texttt{KPOINTS} file must be edited.
Since we are not interested in a uniform sampling of reciprocal space but instead want 
to concentrate on interesting one-dimensional intersections, we must essentially 
explicitly define those in reciprocal coordinates!

Since the exact shape and symmetry of a reciprocal space might be quite hard to 
follow directly from cartesian space if one is inexperienced, the \texttt{xcrysden}
program might be used to analyze it.
Convert the \texttt{POSCAR} file to the xsf-format with the command

\begin{verbatim}
 v2xsf POSCAR
\end{verbatim}

Then, a compressed file \texttt{POSCAR.xsf.gz} will be generated. 
Start the \texttt{xcrysden} program with the command 

\begin{verbatim}
 xcrysden
\end{verbatim}

and open the compressed xsf file with \textit{File $\rightarrow$ Open Structure $\rightarrow$ Open XSF}.
After opening, go to \textit{Tools $\rightarrow$ k-path Selection}. 
In the new window, you can select points of high symmetry. 
Their reciprocal coordinates will directly be written on the right side.
You then might either write them manually or export them to a \texttt{name.kpf} file.

Finally, the \texttt{KPOINTS} file should look something like this:

\begin{verbatim}
 K-points
 30
line mode
fractional
0.0  0.0  0.0   Gam
0.5  0.0  0.0   X

0.5  0.0  0.0   X
0.5 -0.5  0.0   M

0.5 -0.5  0.0   M
0.0  0.0  0.5   Z

0.0  0.0  0.5   Z
0.0  0.0  0.0   Gam
\end{verbatim}

Where two coordinates without empty line are always connected with a one-dimensional path 
for integration. If the paths shall be connected, the coordinate of the first point in the 
second block must be the same as of the second point in the first block.
The number in the second line of the file determines on how many points along one single line the
band energies shall be calculated.

After the calculation is finished, open \texttt{p4v} in the results folder and go to \textit{DOS+bands $\rightarrow$ Show $\rightarrow$ DOS (or DOS and Bands)}.

\section{Bader Charges}

For many purposes, partial charges of the atoms might be useful for evaluation of the 
calculation results. Experimental scientists are often very prone to take partial
charges as central properties of the systems of interest.
In VASP, these charges cannot be calculated directly, since only a part of the electrons 
is described explicitly, whereas others are only mathematical artifacts enshrined into 
the \texttt{POTCAR} files. A simple Mulliken partial charge analysis as in usual HF-based
nonperiodic codes is therefore not possible.
In order to do a Bader charge analysis, the program \texttt{bader} by the Henkelman-group is
needed\footnote{\url{http://theory.cm.utexas.edu/henkelman/code/bader/}}.

First, a VASP calculation must be done. This can be combined with a DOS calculation,
since the requirements for the other keywords are the same as for DOS (simply add them!).
The only special keyword is:

\begin{itemize}
 \item \texttt{LAECHG = .TRUE.} The reconstructed all-electron charge density will be written 
 to the files \texttt{AECCAR0} and \texttt{AECCAR2}.
\end{itemize}

After doing the VASP single-point calculation, evoke the script \texttt{chgsum.pl},
which can be downloaded from the same website as the \texttt{bader} program itself:

\begin{verbatim}
 chgsum.pl AECCAR0 ARCCAR2
\end{verbatim}

Then, calculate the actual bader charges:

\begin{verbatim}
 bader CHGCAR -ref CHGCAR_sum
\end{verbatim}

After finishing, the program as has written two files:

\begin{itemize}
 \item \texttt{ACF.dat} Atom coordinates and Bader charges
 \item \texttt{BCF.dat} Coordinates of Bader maxima
\end{itemize}

For our purpose, only \texttt{ACF.dat} is relevant. 
In it, all atoms of the system are listed. Besides the indices and the Cartesian
coordinates, the column \texttt{CHARGE} contains the actual Bader electron counts.
In order to get the partial charges, the number of explicit electrons for the respective 
element must be subtracted!
These can be obtained by calling:

\begin{verbatim}
 grep ZVAL POTCAR
\end{verbatim}

If, e.g., the \texttt{CHARGE} value of an atom is 3.5 and its \texttt{ZVAL} is 3, its 
partial charge is -0.5, since 0.5 more electrons are located near the atom as it would be 
the case when it is calculated separately.

The calculation of individual charges (and averaged charges for the different involved elements)
can be automatized with the new program \texttt{sum\_bader} (section \ref{sum_bader}).

\section{Core Level (Shifts)}

A property often asked for by experimentalists are energies of core level electrons of 
certain atoms. These can be measured, e.g., by XPS.
Core level energies cannot be extracted from usual VASP calculations (in contrast to 
non-periodic codes like Gaussian), since core-level electrons are not handled 
explicitly within the PAW formalism. 
Instead there are two approaches to calculate core levels: \textit{Initial state} and 
\textit{final state}. 
Both must be done!
In general, for a CLS calculation, the used unit cell should not be too small, which 
should not be a problem for SCALMS or SCILL systems. The number of k-points might
be raised similarly as for DOS calculations.

\subsection{Initial State}

Here, the energies of the core level electrons are calculated in the actual system before 
any excitation (therefore: initial state).
Only add: 

\begin{itemize}
 \item \texttt{ICORELEVEL = 1} activates the initial state core level calculation.
\end{itemize}

After performing a usual single point calculation, open the \texttt{OUTCAR} and search
for:

\begin{verbatim}
  the core state eigenenergies are
     1-  1s -10224.4810  2s  -1257.8857  2p -1094.3492 3s  -144.0040 3p  -94.7441
       3d   -15.0985
....
\end{verbatim}

Here, the core level energies of all orbitals included into the PAW cores of all 
atoms in the system are listed. 
We are interested in the energy of a certain orbital of a certain atom, this value
should be searched for and noted.

\subsection{Final State}

Here, the energy of a certain core level orbital after emitting one electron into the 
valence band and relaxing the whole wave function shall be calculated.
Since only one electron will be excited, it must be specified explicitly.
This can be done with the following keywords:

\begin{itemize}
 \item \texttt{ICORELEVEL = 2}  Activates the final state mode.
 \item \texttt{CLNT = 5} The number of the species which shall be exited. This is somewhat
 complicated: Since only one atom shall be considered, we must somehow choose it. This 
 is done by defining a new species (if more than one atom of the respective element is 
 present). This species will usually be placed at the and of the elements list.
 To do this, we must change \texttt{POSCAR} and \texttt{POTCAR} as well.
 In \texttt{POSCAR}, the lines with the element numbers must be changed, e.g., from
 \begin{verbatim}
     Ni   Ga   H    C
     1   179     4     2
 \end{verbatim}
 to
 \begin{verbatim}
     Ni   Ga   H    C   C
     1   179     4     1   1
 \end{verbatim}
such that we have two carbon species, the last one will be the atom, for which the final
state core level will be calculated.
Since we now have 5 instead of 4 elements in the system, another block must be added 
to the \texttt{POTCAR} file as well! In this case:

       \begin{verbatim}
cat /scratch/potcar/PAW_PBE.52/C/POTCAR >> POTCAR
       \end{verbatim}

  \item \texttt{CLN = 1} The main quantum number N of the electron of interest (here, from the 
  fist shell).
  \item \texttt{CLL = 0} The angular quantum number L of the electron of interest 
  (here, 1s electron) 
  \item \texttt{CLZ = 0.5} The number of electrons to be exited from that level in one 
  unit cell. The value 0.5 seems to be the usual one in practice.
\end{itemize}

After performing the calculation, again open the \texttt{OUTCAR} file and search
for the core level energy line.
The section looks similar to the \texttt{ICORELEVEL = 1} case, all core level
orbitals of all atoms are listed.
Now, however, only the core level energy of the chosen orbital is useful! All others are 
bullshit, which can, e.g., be seen if we choose one of two essentially equivalent 
atoms. The orbital with the same quantum number of the other atom will have 
a totally different (and useless) core level energy.


\subsection{Comparing to Experiment}

We now got two core level energies for the initial and the final state.
Those absolute values, however, are bad and will deviate hugely from experiment!
In the first step, we must convert the energies to the traditional energy scale:
This is done with the formula:

\begin{equation*}
 E_{\text{Corelevel}} = -(E_{\text{Output}}-E_{\text{Fermi}})
\end{equation*}

The Fermi energy can be found by searching in the \texttt{OUTCAR} file:

\begin{verbatim}
 grep "Fermi energy" OUTCAR
\end{verbatim}

The energies for all k-points will be listed, these, however, are always identical.

The resulting core level energies are positive but still not usable for direct comparison
to experiment.
Instead, one usually calculates the core level shifts (differences) in comparison 
to a reference state 
(e.g, to a pure metal bulk or graphene/diamond for Carbon), such that the 
two calculations explained above must be repeated for that reference state.
Then, one can compare the shifts and, e.g., calculate them for different conformers 
or reaction intermediates within the same system.

\section{Simulation of STM Pictures}

For surface science, scanning tunneling microscopy (STM) often serves as a valuable tool. 
Since the electric charge
distribution is monitored, geometric as well as electronic properties can be investigated.
For a STM simulation, two steps need to be made (with the same geometry and other settings):

\begin{enumerate}
 \item Single point calculation of the structure of interest to get the \texttt{CHGCAR} and \texttt{WAVECAR} files of the system.
 (\texttt{LCHARG = .TRUE.} and \texttt{LWAVE = .TRUE.} must be added.)
 \item Calculate the partial charge density which serves as template for simulating the STM picture.
Copy the calculated \texttt{CHGCAR} and \texttt{WAVECAR} file to the folder and add the following keywords:
\begin{itemize}
 \item \texttt{LPARD = .TRUE.} Evaluate partial charge density
 \item \texttt{ISTART = 1} Read in the existent \texttt{WAVECAR} file
 \item \texttt{NBMOD = -3} Calculate partial charge density in an energy range given by \texttt{EINT}.
 \item \texttt{EINT = [value1] [value2]} The energy range for the partial charge density. Usually, \texttt{value1}
 should be set to the negative of the reported experimental STM current to compare with (e.g. -1), \texttt{value2}
 should be zero.
\end{itemize}
\end{enumerate}

The second calculation is only an evaluation of the pre-calculated \texttt{WAVECAR} and should be done within
some seconds to minutes.
After this, the \texttt{PARCHG} file can opened with \texttt{p4v} together with the \texttt{vasprun.xml} file.
In \texttt{p4v}, go to the \textit{STM mode}, switch to \textit{constant current mode} and align the \textit{Isos.density} until
a nice picture can be seen (usually values between 0.01 and 0.03 will be good). This can then be saved by making a screenshot (camera symbol on the left).

For a more quantitative analysis of STM pictures, the screenshots can be opened with the new \texttt{print\_level\_line}
program, 1D cut-troughs can be generated and compared to experimental curves.



\section{Charge Density Difference Plots}

For detailed examination of electronic interactions between surface and adsorbate, it might be useful
to generate a charge density difference plot.
Here, the charge density of the full system is calculated, and then the charge densities of separated 
surface and adsorbate systems are subtracted, such that the effective change in density can be viewed.

First, charge densities of the three systems must be calculated:

\begin{itemize}
 \item Full system
 \item Only surface: delete adsorbate atoms 
 \item Only adsorbate atoms without surface
\end{itemize}

All calculations have to be single points, the positions of the atoms in the pieces must exactly coincide 
those of the full system. 
For this, in all three calculations the following commands must be added: \texttt{IBRION = -1}, \texttt{NSW = 0}
and \texttt{LCHARG = .TRUE.}

After finishing the calculations, open \texttt{VESTA} and then open the \texttt{CHGCAR} file of the full system.
Then, go to \textit{Edit $\rightarrow$ Edit Data $\rightarrow$ Volumetric Data}. There, import the \texttt{CHGCAR} files of the separate systems and
click on \textit{subtract from current data}.

Then, the resulting isosurface can be modified and finally rendered by: \textit{File $\rightarrow$ Export $\rightarrow$ Raster Image}.


\section{Two-Dimensional Charge Densities}

Charge densities can also be evaluated directly by looking at 2D charge densities.
For this, only the full system needs to be calculated.
The settings are similar to those for DOS calculations:

\begin{itemize}
 \item \texttt{LCHARG = .TRUE.} Write out the charge density
 \item \texttt{LPARD = .TRUE.} Do a partial charge density calculation
 \item \texttt{NBMOD = -3} Energy range for chosen bands will be given in eV
 \item \texttt{EINT = -4.0  0.0} The energy range in eV for the bands, which will be integrated
 to get the charge density within this energy range (with respect to the Fermi level).
 Here, the integration will be done from -4 eV to 0 eV with respect to the Fermi level.
\end{itemize}

After the calculation has been finished, copy the \texttt{CHGCAR} file to a suitable
folder and open it with \texttt{VESTA}.

On the molecular viewer, select three atoms (with the angle selection tool) which are
located on the plane that shall be looked at with a 2D density plot.
If only two atoms are located on the plane, it is a good strategy to widen the
view to two unit cells in z-direction with the boundary tool and select one
of the periodic copies of one atom as the third atom.

Then, go to \textit{Utilities $\rightarrow$ 2D Data Display} and click on the \textit{Slice}
button in the new window. There, click on \textit{Calculate the best plane for the
selected atoms}. Now, a new window should open which shows a color-gradient density
plot (blue: low density, red: high density).
The color scale can be changed with the \textit{Saturation Levels} toolbox.
Further, \textit{Contour Lines} can be drawn by activating them in the
\textit{Contours} category. There, a scale of maximum and minimum density
values can be set as well.

If everything looks good, go to \textit{File $\rightarrow$ Export Raster Image}.


\section{Nudged Elastic Band}

One central challenge in theoretical chemistry is the parametrization of reaction paths.
In the simplest picture, one needs reactants, transition state and products in order to
describe the central properties of a reaction.
The search for transition states, however, is one of the most challenging tasks one 
can think of in the VASP framework.
There is no direct transition state optimization routine implemented, in contrast to, e.g., 
Gaussian, instead, one needs to optimize a whole reaction path with the Nudged Elastic 
Band (NEB) method, which can be modified to include the optimization of the point
of highest energy along the path -- the transition state.

Since the preimplemented NEB tool in VASP has limited features, an external program 
update called \textit{VTST for VASP}\footnote{http://theory.cm.utexas.edu/vtsttools/} should be
added to the program version.
The installation manual can be found on the website.
The VTST scripts on the same website are needed as well. Copy them to a folder being in the 
\texttt{PATH} variables.

For the optimization of a TS (and the reaction path), the following steps should be done:

\begin{enumerate}
 \item Optimize the structures of reactant and product. Generate initial guesses for the structures 
 and optimize them with the methods explained above.
 \item Copy the \texttt{CONTCAR} files of reactant and product and decide, how many frames the 
 reaction path between the endpoints shall have. The more frames, the better the convergence of the NEB might be, but the 
 calculation effort scales linearly with this number! A good compromise might be 8 or 16 structures.
 For 8 structures, copy the \texttt{CONTCAR} of the reactant to \texttt{POSCAR00} and that of the product 
 to \texttt{POSCAR09}. 
 \item Execute the script \texttt{nebmake.pl} from the VTST scripts folder. Now, 10 folders called 
 \texttt{00}, \texttt{01}, ... \texttt{09} are generated, in which \texttt{POSCAR} files of the linearly interpolated 
 initial structures of the reaction path segments are located.
 \item In the main folder containing the newly generated folders, place the usual VASP input files 
 (but no \texttt{POSCAR}).
 The \texttt{INCAR} must contain the following additional keywords:
 \begin{itemize}
  \item \texttt{IMAGES = 8} The number of NEB images (reaction path segments) \textit{between} the 
  fixed endpoints.
  \item \texttt{SPRING = -5} The spring constant between two images of the path. For NEB, use 
  a negative value.
  \item \texttt{ICHAIN = 0} Activates the NEB method.
  \item \texttt{LCLIMB = .TRUE.} Activates the climbing image method, i.e., the targeted 
  optimization of the transition state along the path.
  \item \texttt{LNEBCELL = .FALSE.} Turns on SS-NEB. (Used with ISIF=3, i.e., optimization of the 
  cell size/volume, not relevant for us!)
  \item \texttt{IOPT = 3} Uses a special optimization routine from the Henkelman group (\textit{Quick-Min}) for the 
  automatic choice of \texttt{IBRION=1} or \texttt{IBRION=3} (force-based optimizer).
  \item \texttt{IBRION = 3} Disable VASP default optimizers for \texttt{IOPT = 3}.
  \item \texttt{POTIM = 0} Sets the VASP geometry optimization step to zero such that all the 
  work is done by the addon.
  \item \texttt{TIMESTEP = 0.1} Dynamical time step (?), might not be needed at all...
 \end{itemize}
\item Start the calculation, note that the total number of processors must be divisible through the 
number of images! During the calculation, the status can be checked by invoking the \texttt{eval\_neb.py}
script in the main folder (see below).
\item After converging (or running out of walltime), look for the structure with the highest energy and 
check if it is really a transition state: Copy the \texttt{CONTCAR} file in the respective folder 
into a new one and do a frequency calculation. Only one (large) imaginary mode should appear, the 
corresponding normal mode vector should somehow resemble the desired reaction.
\end{enumerate}

\section{TS Optimization with the Dimer Method}

After having found a good initial guess for a transition state (either from a NEB calculation
or other methods like making an educated guess), the structure should be refined by a separate
TS optimization procedure. The standard implementation in VASP (if VTST is compiled with it, see
previous section) for this means is the dimer method.

In the \texttt{INCAR} file, the following changes need to be made:

\begin{itemize}
 \item \texttt{ICHAIN = 2} Activates the dimer method
 \item \texttt{IBRION = 3} MD with zero time step
 \item \texttt{POTIM = 0} Zero time step, ions are not moved 
 \item \texttt{DdR = 0.005} The dimer separation, twice the distance between the images 
 \item \texttt{DRotMax = 4} Number of rotation steps per dimer translation
 \item \texttt{DFNMin = 0.01} Lower threshold for dimer rotation (rotational force)
 \item \texttt{DFNMax = 1.0} Upper threshold for dimer rotation
 \end{itemize}

Aside these settings, the \texttt{POSCAR} file needs to include the initial guess for the TS structure.
 
Further, a file named \texttt{MODECAR} must be generated. In this file, the initial direction along the 
imaginary mode, where the dimer is started, must be given. If a frequency calculation on the TS initial
guess was done beforehand, the \texttt{OUTCAR} file from this calculation must be opened.
Go to the end of the file and search from below the section in which the normal modes of the 
vibrations are listed. Take the mode with the largest imaginary frequency (or the one, which seems 
to be the most reasonable after looking at it with \texttt{gmolden}) and copy the right part with the 
displacements, for example:

\begin{verbatim}
  18 f/i=   15.881852 THz    99.788616 2PiTHz  529.761543 cm-1    65.682031 meV
             X         Y         Z           dx          dy          dz
      9.057219 12.556697 15.441843     0.016824    0.007018   -0.000929
     11.033089 14.113899 16.148915    -0.021587    0.001532   -0.008405
     11.070764 13.777563 14.308820    -0.021653    0.001840    0.004560
     11.395509 12.363126 15.525165     0.230301    0.074750   -0.007783
     11.124066 13.414659 15.327396    -0.260907   -0.044059    0.002170
     13.334577 13.413904 15.372707     0.045110   -0.003492    0.001647
\end{verbatim}
The last three columns without headers must be copied to the \texttt{MODECAR} file, such that it has 
the following content:

\begin{verbatim}
   0.016824    0.007018   -0.000929
  -0.021587    0.001532   -0.008405
  -0.021653    0.001840    0.004560
   0.230301    0.074750   -0.007783
  -0.260907   -0.044059    0.002170
   0.045110   -0.003492    0.001647
\end{verbatim}

Now, the dimer TS optimization can be started. 
The resulting \texttt{CONTCAR} file might again be tested by doing a frequency calculation with it.


\section{Steered Molecular Dynamics}

For complex reaction mechanisms with sterical hindrance is might often be complicated if not
impossible to find a reasonable reaction path by a NEB calculation.
In this cases, biased molecular dynamics could be the tool of choice to find a first guess
for the reactive process and its intermediates/transition states.

Since exhaustive samplings of reaction paths by combined umbrella samplings needs tens of
thousands of single energy/gradient calculations, this is usually too expansive for
AIMD. Steered MD, on the other hand, is much cheaper.
In this case, a \textit{time-dependent umbrella potential} is applied to the coordinates
of interest, which drives the system from reactant to product state within 1-2 ps.
Unfortunately, there is no existent official VASP documentation for umbrella sampling and
steered MD, the following settings were obtained by looking through the VASP-code directly...

In the \texttt{INCAR} file for a usual AIMD calculation, the following additions need to
be made:

\begin{itemize}
 \item \texttt{MDALGO = 2} The Nose-Hoover thermostat should be activated.
 \item \texttt{SPRING\_K} Strength of the force constant of the harmonic umbrella potential (in eV\AA). The exact value depends
 on the system and the desired length of the AIMD trajectory. For short trajectories and high potential
 energy barriers along the reaction path, higher values might be used. If the values are too large, however,
 the molecule might be torn apart. A good range seems to be between 2 and 6.
 For each coordinate to be biased, an extra number need to be added in the same line!
 \item \texttt{SPRING\_R0} The position of the umbrella potential with respect to the chosen coordinate (e.g., the bond
 length, at which no retraction force is applied). In the case of steered MD, this should be the actual value(s)
 of the coordinate(s) in the \texttt{POSCAR} file. In general, again an array with one value per biased coordinate.
 \item \texttt{SPRING\_V0} The movement speed of the umbrella potential position in \AA/fs (distances) or rad/fs (angles).
 The value should be precalculated such that within the given number of simulation steps (\texttt{NSW}), the desired
 final values of the biased coordinates are reached. Positive values will enlarge the value of the chosen coordinate,
 whereas negative values will shrink it. Again, one value per biased coordinate must be given.
\end{itemize}

Additionally, a separate input file named \texttt{ICONST} containing the restraints and constraints within the
system must be given (else, all Spring keywords in the \texttt{INCAR} will be ignored).
In this file, all coordinates on which umbrella potentials shall be applied must be given.
If, e.g., an umbrella potential shall be applied on the distance between the atoms \texttt{3} and \texttt{4}, the following
line is sufficient:

\begin{verbatim}
 R 3 4 8
\end{verbatim}

The last number (\texttt{8}), also called the ``\texttt{STATUS flag}'', specifies the application of a harmonic potential.
For two separate coordinates steered during the same trajectory, two lines are needed:

\begin{verbatim}
 R 11 12 8
 R 13 14 8
\end{verbatim}

Alternatively, one can define composite coordinates like the distance between the center of masses of two bonds
in the system.
If one bond exists between atoms \texttt{11} and \texttt{13} and another between atoms \texttt{12} and \texttt{14},
the composite coordinate would be the following:

\begin{verbatim}
 B 11 13 12 14 8
\end{verbatim}

More details on coordinate definitions can be found in the VASP wiki (\url{https://www.vasp.at/wiki/index.php/ICONST}).

During a steered MD simulation or after it, the current status can be evaluated by invoking the utility
program \texttt{eval\_steered} (section \ref{eval_steered}).


\section{Utility Scripts}

In the last year, some utility scripts were written (or obtained by Christian Neiß) which might be useful at
some point. 
Warning: Some of the scripts need \texttt{POSCAR}s with either direct or Cartesian
coordinates! This will be automated soon, currently, however, the input files 
should be converted with \texttt{p4v} if the results are nonsense.
All general or CLINT related scripts are listed here:

\subsection{xyz2poscar.py}\label{xyz2poscar}

It is often more easy to click together a simple xyz file of a molecule
with Avogadro or similar 
programs. Such a structure can then be converted into the VASP format with 
\texttt{xyz2poscar.py}.

The usage is:

\begin{verbatim}
 xyz2poscar.py [xyz-file] POSCAR
\end{verbatim}

By default, the molecule is placed in a cubic box of 20 Angstroms edge length.
This might be changed by altering the \texttt{boxdim} variable in the 
header section of the script.

\subsection{poscar2xyz.py}\label{poscar2xyz}

Converts a POSCAR file back to a xyz file (named \texttt{output.xyz}).

The usage is simply:

\begin{verbatim}
 poscar2xyz.py 
\end{verbatim}

\subsection{contcar2xyz.py}\label{contcar2xyz}

Similar to \texttt{poscar2xyz.py}, here a CONTCAR file is converted
to \texttt{output.xyz}:

\begin{verbatim}
 contcar2xyz.py
\end{verbatim}

Further, single atoms that were shifted on the other side of the cell due to
periodic boundary conditions are moved on the former side, if they are close
to the edge, e.g., for slab calculations where single surface atoms might else appear
on the upper side of the vacuum.

\subsection{build\_scalms.py}\label{build_scalms}

Builds up a SCALMS random alloy containing two or three different elements.
The atoms will be placed on a regular grid, where the elements are
chosen by chance.
After calling the script in the command line, the user is asked if two or
three different elements are used, then for the symbols of the involved
elements and finally for the number of atoms for each element.

Further settings are: Number of atoms in x and y direction, distance between
two atoms and size of the vacuum in z-direction, if a surface slab shall be build.
These variables are coded in the header of the python script file and can be
changed there. Note that the script will build up the regular structure layer by layer,
therefore, one should choose a total number of atoms divisible by the number of atoms 
per layer if a bulk system shall be build up (then, the vacuum is z-direction must be set to zero in 
the script or in the resulting  \texttt{POSCAR} file).

Then, a \texttt{POSCAR} file is written.

Before using the resulting structures for actual calculations, they should be 
equilibrated with AIMD (or ML-MD) to a liquid slab (or bulk) system using the 
settings for molecular dynamics explained above.

\subsection{build\_adsorbates.py}\label{build_adsorbates}

Places molecules on a given surface. This script can be used to set up
SCILL adsorption patterns as well as input for reactions on SCALMS systems or others.
Before invoking the script, a folder containing the preoptimized structures
of the available adsorbates must be given in the header of the script:

\begin{verbatim}
 mol_path="/home/jsteffen/work/build_adsorbates_input/"
\end{verbatim}

In this folder, each species must have a subfolder, whose name must be defined
in the script as well. In each subfolder, a POSCAR file of the species must be
located.

In addition, a folder containing the VASP \texttt{POTCAR} files must be linked
in the script header, if \texttt{POTCAR} files for the adsorbate structures
shall be written as well:

\begin{verbatim}
 potcar_path = "/scratch/potcar/PAW_PBE.52/"
\end{verbatim}

Now, the actual script handling is described:

Two input files must be given:
\begin{enumerate}
 \item \texttt{POSCAR\_surf} The structure of the surface slab, on which the
 adsorbates shall be placed. The structure must be given in Cartesian coordinates.
 \item \texttt{adsorb\_list.txt} Which adsorbates shall be placed on which position,
 and how their rotation shall be.
 A simple example would be:
 \begin{verbatim}
  99 4.5 7.0 2.2 0.0 90.0 0.0
 \end{verbatim}
 Here, species 99 (which always is a structure located directly in the folder, where
 the script is executed, named \texttt{POSCAR\_ads}) shall be placed on a position
 4.5 \AA shifted along the first axis and 7.0 \AA shifted along the second axis
 from the origin. It shall be placed 2.2 \AA above the surface (the lowest atom of the
 molecule will be 2.2 \AA above the highest atom of the surface).
 The first Euler angle is 0 degree, the second 90 degree and the third again 0 degree.
 This can be generalized to an arbitrary number of species, where one species is
 defined in one line. The indices are defined by the script (looking at the list of
 available species located in \texttt{mol\_path}), where 99 is the
 special case of an arbitrary molecule or structure stored in \texttt{POSCAR\_ads}.
 \end{enumerate}

 If all files are prepared, execute

 \begin{verbatim}
  build_adsorbates.py
 \end{verbatim}

 and a \texttt{POSCAR} file containing the adsorbate on the surface will be printed out
 together with the \texttt{POTCAR} file of the system.
 The actual placement of the species can be quite tedious, especially if non-orthogonal
 unit cells are used. The process should be done by iteratively executing the script
 and opening the resulting \texttt{POSCAR} file with \texttt{VESTA} and decide by clicking
 at surface molecules, which coordinates the center of masses of the adsorbates shall have.




\subsection{shift\_unitcell.py}\label{shift_unitcell}

This script is able to shift all atoms in the unit cell by fractions of the 
unit cell vectors. If, e.g., the catalytic center is located near the edge of the 
cell, it might be visually more attractive to have it in the middle.
This is of course a purely aesthetical correction, since the periodicity of the 
system stays the same.

Give a \texttt{POSCAR} file in direct coordinates and start the script.
It is useful to inspect the system first with \texttt{VESTA} or \texttt{p4v} 
and determine the ideal shift vectors (with x,y,z components from 0.0 to 1.0).

After finishing, a file \texttt{POSCAR\_shifted} is written.

\subsection{cut\_unitcell}\label{cut_unitcell}

This Fortran program enables the generation of almost arbitrarily inclined surface 
unit cells, needed, e.g., for SCILL adsorption patterns.
Only requirement is that the edged of the produced unit cell must always be located on 
surface atoms.
Two input files are needed to do the job:

\begin{itemize}
 \item \texttt{POSCAR\_surf} Containing a large reference cell of the metal slab, e.g., generated by 
 multiplying a unit cell with \texttt{p4v} 20 times in a and b direction. This slab should 
 be large enough to fully contain the desired inclined unit cell.
 \item \texttt{atom\_inds.txt} Three lines with one atom index each, marking the edges of the 
 inclined unit cell. The first index is the origin in the lower left, the second index is the 
 end point of the a vector, the third index the end point of the b vector.
In order to let the program work properly, atoms 2 and 3 should always be located to the right and above 
the first atom!
\end{itemize}

After execution, the resulting cut/inclined cell is written to the file \texttt{POSCAR}.
Sometimes it might happen that atoms within this cut cell are doubled at the same position (future update needed!).
Then, a VASP single point calculation might be done where the pair-distances are written directly 
at the beginning of the \texttt{OUTCAR} file. Remove all second atoms for distances of (almost) 0 \AA~in 
the \texttt{POSCAR} file!


\subsection{xdat2xyz.py}\label{xdat2xyz}

This script converts a \texttt{XDATCAR} file from a geometry optimization or a MD
calculation to a xyz file named \texttt{trajectory.xyz} which can then be 
viewed with molecular viewers like \texttt{jmol} or \texttt{vmd}.

\subsection{pick\_xdat.py}\label{pick_xdat}

If a certain structure along a MD trajectory or an optimization is of certain
interest and is needed for subsequent evaluation calculations, this can be 
picked from the \texttt{XDATCAR} file by invoking:

\begin{verbatim}
 pick_xdat.py [Name of XDATCAR] [Structure No.]
 \end{verbatim}

 A file \texttt{POSCAR\_XDAT} is written. Note that if selective dynamics is needed for 
 subsequent calculations, the respective keyword and flags must be added manually, since \texttt{XDATCAR}
 structures never include such information!
 
\subsection{vasp\_long.sh}\label{vasp_long}

If a geometry optimization or a MD calculation is started on a cluster of the 
Rechenzentrum, the maximum walltime of one day might be insufficient for completing 
the calculation. In order to avoid manual restarting every 24 hours, this script 
automates this job.

Simply place the usual input for the calculation in a folder and start the 
script with:

\begin{verbatim}
 vasp_long.sh [number of restarts] &
\end{verbatim}

Note: Sometimes the script is killed when the ssh session in which it was started 
is closed. Therefore, don't close the window! (or use \texttt{tmux}).

During the calculation, the \texttt{OUTCAR} and \texttt{XDATCAR} files 
are copied and indices are added in order to archive them: \texttt{OUTCAR1},
\texttt{OUTCAR2}, ...


\subsection{ml\_long.sh}\label{ml_long}

Starts an arbitrary long on-the-fly machine learning trajectory.
This script can be started in a folder containing the usual input for a 
one-part machine learning (see section in the main part).
The number of time steps \texttt{NSW} should be set to a number that high
that convergence of the learning can be expected. The script will restart the 
learning until all steps in the \texttt{NSW} keyword are calculated.

Start the script with 

\begin{verbatim}
 ml_long.sh [number of parts] &
\end{verbatim}

To be sure that the script will not be killed, start the calculation in a \texttt{tmux} window.

During the calculation, the \texttt{OUTCAR}, \texttt{XDATCAR}. \texttt{CONTCAR}, \texttt{ML\_ABN} 
and \texttt{ML\_LOGFILE} files of each part are renamed to e.g. \texttt{XDATCAR1} etc., such that 
analysis/debugging can be done based on them. 

In addition to restart the parts, several annoying VASP/slurm errors (\textit{No space left on 
device}, \textit{srun: Job step aborted}. \textit{Firewal refused connection} or \textit{NaN} 
machine learning errors) are being looked for. If they appear, the calculation is restarted 
(and aborted, if needed) automatically.

The script stops either if all parts were calculated (each terminated by the cluster 
for exceeding the maximum walltime) or if the desired total number of \texttt{NSW} steps 
is finally calculated.

The actual status of the overall calculation can be inspected by looking into \texttt{ml\_long.log}.

In order to be able to restart the calculation completely, the \texttt{INCAR} file of the first 
part (given by the user) is copied to \texttt{INCAR\_initial}.

\subsection{md\_long.sh}\label{md_long}

Starts an arbitrary long molecular dynamics trajectory (ab initio or machine learning).
The general setup is very similar to \texttt{ml\_long.sh}.
Input files for an MD must be given, the number of steps (\texttt{NSW}) now should be set to the number 
one wants to calculate, no matter, how long such a calculation will take.
The script will restart the calculation until the total MD step number has been calculated. 

The start is again:

\begin{verbatim}
 md_long.sh [number of parts] &
\end{verbatim}

and might be done with \texttt{tmux.}
Output of each part is renamed (\texttt{OUTCAR}, \texttt{XDATCAR}, \texttt{CONTCAR}) and stored, 
actual status is updated in \texttt{md\_long.log}.

\subsection{contcar2poscar}\label{contcar2poscar}
 
This Fortran program converts a \texttt{CONTCAR} file to a \texttt{POSCAR} file 
and removes all periodic image flags. 
VASP is quite stupid in this regard and allows only two leading digits in the 
positions. If during a long MD run (which can readily happen during ML-MD), more than 9 periodic cell repetitions 
are crossed by an atom, the position cannot be written anymore.
This script automatically cleans up such nasty problems and stores \texttt{POSCAR}s with 
direct coordinates between 0.0 and 1.0.
This script is called by \texttt{vasp\_long.sh}.

\subsection{grad2}\label{grad2}

This Python script monitors geometry optimizations and dynamic calculations (by Peter 
Larsson\footnote{\url{https://www.nsc.liu.se/~pla/vasptools/}}).
Simply invoke:

\begin{verbatim}
 grad2 OUTCAR
\end{verbatim}

and the actual status will be written out.


\subsection{grad2select}\label{grad2select}

A slightly changed version of \texttt{grad2}. If a geometry optimization with \textit{selective dynamics}
is done, the \texttt{grad2select} script is of little use if one wants to check, how far the 
system is still away from convergence, if the gradient criterion is used, since 
the largest of all gradient components is always shown, no matter if the respective 
atom is activated for optimization or not.
\texttt{grad2select} directly shows the largest gradient component of the subset of atoms 
activated for optimization.
Further, if convergence was not reached so far, the indices of all not converged atoms are printed 
out as well as the atom on which the largest force acts.
Invoke:

\begin{verbatim}
 grad2select OUTCAR
\end{verbatim}

Besides the command line output, the file \texttt{grad2select.log} is written in which
the energy, average force and maximum force is printed, such that the 
convergence behavior can be further investigated by plotting those values.

\subsection{eval\_vasp\_md}\label{eval_vasp_md}

This Fortran program tries to imitate the now deprecated \texttt{nMOLDYN} 
program\footnote{\url{https://github.com/khinsen/nMOLDYN3/}} for analysis
of MD trajectories.
Currently, it is focused on the calculation of \textit{diffusion coefficients} 
of bulk SCALMS systems (up to three components).

For such an analysis, a molecular dynamics (AIMD or ML-MD) trajectory
of a periodic bulk system (no surface slab) must be calculated beforehand.
In the folder where the resulting \texttt{XDATCAR} trajectory file
is located, the program can be executed.

It can be controlled by command line flags:

\begin{itemize}
 \item \texttt{-msd} The diffusion coefficient is calculated by
 mean square displacement (MSD) analysis.
 \item \texttt{-vacf} The diffusion coefficient is calculated
 by the velocity autocorrelation function (VACF) analysis.
 \item \texttt{-dt=...} The time step of the MD calculation in fs.
 \item \texttt{-skip=...} The number of MD time steps at the beginning
 of the trajectory that shall be skipped for further analysis.
 \item \texttt{-npt} If a NpT trajectory with variable volume was
 simulated, this flag should be added for analysis such that the
 actual volume in each MD time step is recalculated. The simulation
 box still needs to be of orthogonal shape!
\end{itemize}

After execution, the calculated diffusion coefficients of all
one to three elements are written to the files \texttt{diff\_const\_MSD\_elx.dat} or
\texttt{diff\_const\_VACF\_elx.dat}, depending whether the MSD
or VACF analysis was done. Further, time dependent plots of the
MSD or (integrated) VACF functions are written.


\subsection{analyze\_scalms}\label{analyze_scalms}

This Fortran programs analyzes long MD trajectories of SCALMS random
alloys containing two or three components (where the last component is assumed
to be the active metal).
The whole dynamics is read in from a \texttt{XDATCAR} file.
During the analysis, element distributions along the z-axis are 
calculated, example structures of the less abundant atom in different 
slices of the slab are printed out and radial distribution functions 
are calculated (if desired).
Further, time-dependent z-coordinates of active atoms are 
plotted.

The trajectory within \texttt{XDATCAR} is converted to the xyz format
and saved in the file \texttt{trajectory.xyz}.
Further, the element densities are written to the file \texttt{dens\_elems.dat}.
Time-dependent positions of all active atoms are written to the file
\texttt{active\_positions.dat}.
Further, example structures (\texttt{POSCAR} format) containing an active atoms in a number of
different slices along the c-axis are stored within the folder
\texttt{core\_levels} and can be used, e.g., for subsequent
core level shift calculations.

Add the flag \texttt{-3elems} if ternary SCALMS shall be calculated.
The flag \texttt{-notraj} deactivates the printout of \texttt{trajectory.xyz}, which
might be useful for long and huge trajectories, whereas the flag \texttt{-rdf} activates
the printout of radial distribution function (only for binary SCALMS).

\subsection{calc\_bader.py}\label{calc_bader}

Calculates (evaluates) the effective Bader partial charge of an atom.
When the VASP calculation with the needed settings was done, invoke this 
script to automatize all following steps described above.
All what must be done is to change the settings in the header section 
of the script.
Especially useful for larger series of Bader charge calculations (in order 
to get some statistics).

\subsection{sum\_bader}\label{sum_bader}

An advanced version of \texttt{calc\_bader.py}. Here, the \texttt{POSCAR} 
and \texttt{ACF.dat} files of a Bader charge calculation must be present as input.
The program calculates the Bader partial charge (corrected by the PAW charge)
for each atom in the system and averages the charges of all involved elements.

After executing it by simply typing

\begin{verbatim}
 sum_bader
\end{verbatim}

the averaged charges are printed to the comment line.
The charges for each element are printed to the file \texttt{bader\_charges.dat} and might 
be further analyzed or, e.g., visualized by VMD after adding them as separate column
to a xyz file.

\subsection{eval\_steered}\label{eval_steered}

This Fortran program extracts the current status of a steered MD simulation by
evaluating the file \texttt{REPORT}. The results are written to the file
\texttt{steered.dat}. Within it, each line corresponds to one MD step.
Listed are the actual values and umbrella potential positions of biased coordinates,
potential, kinetic and total energy of the system as well as ideal and real temperature.
Those values can be plotted using, e.g., \texttt{gnuplot}. Up to 4 biased coordinates
can be analyzed by the program.

\subsection{calc\_field\_co}\label{calc_field_co}

This Fortran program evaluates electric field changes introduced by external electric fields.
The current application area of the program is quite limited: calculate averaged electric fields
along the bond axis of a CO molecule adsorbed on a metal surface (in order to compare it
with CO frequency Stark shifts measured by experiment).

It needs to \texttt{LOCPOT} files of the calculations with and without external electric field
(\texttt{LOCPOT\_on} and \texttt{LOCPOT\_off}).
The subtraction of both \texttt{LOCPOT} files is done automatically, as well as the
evaluation of numerical derivatives along the axis of the adsorbed CO.

Even if such local electric fields of other systems shall be calculated, this program might
serve as a viable starting point for the implementation of such analysis.

\subsection{vasp2molden.py}\label{vasp2molden}

Converts the results of a VASP frequency calculation (frequencies, normal modes) into the 
more useful molden format and calculates the IR intensities of the vibrations, if the frequency 
calculation was done with dipol correction (by Christian Neiß).

Simply copy the \texttt{vasprun.xml} file into the analysis folder and execute 

\begin{verbatim}
 vasp2molden.py
\end{verbatim}

Then, the frequencies and intensities are written to the command line; further, a file 
\texttt{vasprun.molden} is generated which can be opened by \texttt{gmolden} in order to
look at the normal modes or to plot the IR spectrum.

\subsection{vasp\_cellparam.py}\label{vasp_cellparam}

Analyzes the properties of the unit cell of the system (by Christian Neiß).

\subsection{vasp\_gap.py}\label{vasp_gap}

Gives a comprehensive overview of the results of a VASP calculation by 
reading in the \texttt{vasprun.xml} file. Several calculation settings 
are given, as well as the final energy, orbital occupations and HOMO/LUMO
gaps (based on orbital occupations and Fermi energy) (by Christian Neiß).


\section{Calculation Clusters}

Almost call calculations with larger systems will be done on external calculation
clusters.
Currently, there are four branches available:

\begin{itemize}
 \item Local Cluster (\texttt{tcsv020})
 \item RRZE Clusters (\texttt{emmy}, \texttt{meggie}, \texttt{woody}).
 \item The Fritz Cluster (currently, restricted access)
 \item The CLINT Cluster (new, no working VASP with MPI so far)
\end{itemize}

All but \texttt{emmy} and \texttt{woody} use the modern \texttt{Slurm}
batch management software.
With this, VASP calculation jobs can be submitted to the large cluster machine.




\end{document}
